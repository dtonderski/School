
% This LaTeX was auto-generated from MATLAB code.
% To make changes, update the MATLAB code and republish this document.

\documentclass{article}
\usepackage{graphicx}
\usepackage{color}

\sloppy
\definecolor{lightgray}{gray}{0.5}
\setlength{\parindent}{0pt}

\begin{document}

    
    \begin{verbatim}
function E2=PAS(E1,L,N,a,lambda_noll,n_medium)

% Varje sampelpunkt i k-planet motsvarar en plan v�g med en viss riktning (kx,ky,kz)
delta_k=2*pi/(N*a); % samplingsavst�nd i k-planet *** Klar
kxvekt=-N/2*delta_k:delta_k:(N/2-1)*delta_k; % vektor med sampelpositioner i kx-led
kyvekt=kxvekt; % och ky-led
[kxmat,kymat]=meshgrid(kxvekt,kyvekt); % k-vektorns x- resp y-komponent i varje sampelpunkt i k-planet

k=2*pi*n_medium/lambda_noll; % k-vektorns l�ngd (skal�r) f�r en plan v�g i ett material med brytningsindex n_medium *** Klar
kzmat=sqrt(k^2-kxmat.^2-kymat.^2); % k-vektorns z-komponent i varje sampelpunkt i k-planet *** Klar (Obs! Matlab till�ter att en skal�r direkt adderas/subtraheras med matris, man beh�ver allts� tex inte skriva "skal�r*ones(N,N)-matris")

fasfaktor_propagation=exp(1i* kzmat*L); % faktorn varje sampelpunkt i k-planet (som ju motsvarar plan v�g i viss riktning) multas med f�r att propagera str�ckan L i z-led *** Klar

A= a^2/(2*pi)^2*fft2c(E1); % Planv�gsspektrum i Plan 1 *** Klar

B=A.*fasfaktor_propagation; % Planv�gsspektrum i Plan 2 (Planv�gsspektrum i Plan 1 multat med fasfaktorn f�r propagation f�r varje plan v�g)

E2= delta_k^2*N^2 *ifft2c(B); % *** Klar
\end{verbatim}



\end{document}
    
