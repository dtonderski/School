\documentclass{article}
\renewcommand{\thesubsection}{\thesection.\alph{subsection}}
\usepackage{siunitx}
\usepackage{fancyvrb}
\usepackage{graphicx}
\usepackage[final]{pdfpages}
\setboolean{@twoside}{false}

\fvset{tabsize=4}
\begin{document}
	\title{Hupp2}
	\author{David Tonderski\\davton\\davton@student.chalmers.se}
	\date{}
	\maketitle
	
\setcounter{subsection}{1}
\setcounter{section}{1}
\subsection{}
Koden bifogas i avsnitt 2.b.
\subsection{}

Koden bifogas i avsnitt 2.c.\\\newline
\indent{}\textbf{Fall i)}\\
\indent{}$E_{ut}=(i,0), I_{ut}=1$\\\newline
\indent{}\textbf{Fall ii)}\\
\indent{}$E_{ut}=10^{-15}(-0.2+0.05i,0), I_{ut}=5\cdot10^{-32}$\\\newline
\indent{}\textbf{Fall iii}\\
\indent{}$E_{ut}=10^{-15}(-0.2+0.05i,0), I_{ut}=5\cdot10^{-32}$\\\newline
\indent{}\textbf{Fall iv)}\\
\indent{}$E_{ut}=(i,0), I_{ut}=1$
\subsection{}
Koden bifogas i avsnitt 2.d.\\\newline
Intensiteten på spökbilden blir ungefär 0.15, alltså 15\% av det ingående fältets intensitet.
\subsection{}
Koden bifogas i avsnitt 2.e.\\\newline
Om man vrider glasen vid papperslapparna \SI{90}{\degree} så blir den oönskade bildens intensitet ca. $7 \cdot 10^{-32}$ (den försvinner). Priset man får betala är att den önskade bildens intensitet minskar till ca. 0.85 av det utgående fältets densitet.

\section{MATLAB}
\setcounter{subsection}{1}
\subsection{}
\textbf{J\_pol.m}
    \begin{Verbatim}
function matris = J_pol( alfa )
	polarisationsmatris=[1 0; 0 0];
	matris = J_proj(-alfa) * polarisationsmatris * J_proj(alfa);
end
\end{Verbatim}
\textbf{J\_ret.m}
    \begin{Verbatim}
function matris = J_ret(alfa, phi)
	retarderingsmatris = [exp(j*phi) 0; 0 1];
	matris = J_proj(-alfa)*retarderingsmatris*J_proj(alfa);
end
\end{Verbatim}
\subsection{Hupp2c.m}
Koden finns på nästa sida.

\includepdf[pages=-]{hupp2c.pdf}
\subsection{Hupp2d.m}
Koden finns på nästa sida.

\includepdf[pages=-]{hupp2d.pdf}
\subsection{Hupp2e.m}
Koden finns på nästa sida.
\includepdf[pages=-]{hupp2e.pdf}





\end{document}